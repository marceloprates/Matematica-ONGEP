
\documentclass[usenames,dvipsnames,svgnames]{beamer}

\usepackage[utf8]{inputenc}
\usepackage{lmodern}
\usepackage{array} % needed for \arraybackslash
\usepackage{graphicx}
\usepackage{adjustbox} % for \adjincludegraphics
\usepackage{amsmath}
\usepackage{tikz}
\usepackage{tkz-euclide}
\usepackage[portuguese]{babel}
\usepackage{pgfplots}
\usepackage{adjustbox}

%Information to be included in the title page:
\title{Funções Lineares, Quadráticas e Modulares}
\author{Matemática}
\institute{ONGEP}
\date{2018}

\begin{document}

\frame{\titlepage}

\begin{frame}	
	\frametitle{Funções Injetoras, Sobrejetoras e Bijetoras}

	\begin{itemize}
		\item Uma função injetora (ou injetiva) é uma onde cada elemento diferente do domínio mapeia para um elemento diferente da imagem: $a \neq b \Rightarrow f(a) \neq f(b)$
		\item OBS. toda função função injetora é \textbf{invertível à esquerda} (e vice-versa). Por quê?
	\end{itemize}

	\begin{adjustbox}{max totalsize={\textwidth}{.65\textheight},center}
	\begin{tikzpicture}
	\centering
	\begin{axis}[
		    axis lines = left,
		    xlabel = $x$,
		    ylabel = {$f(x)$},
		    legend style={at={(1,0.125)},anchor=south east}
		]
		\addplot [
		    domain=-3:3, 
		    samples=100, 
		    color=blue,
		    ]
		    {x^3};
		\addlegendentry{$x^3$}
		\addplot [
		    domain=-3:3,
		    samples=100, 
		    color=red,
		]
		{4*x^2 - 5};
		\addlegendentry{$4x^2 - 5$}
	\end{axis}
	\end{tikzpicture}
	\end{adjustbox}

\end{frame}

\begin{frame}	
	\frametitle{Funções Injetoras, Sobrejetoras e Bijetoras}

	\begin{itemize}
		\item Uma função sobrejetora (ou sobrejetiva) é uma onde todo elemento no codomínio é ``atingido'' (a imagem da função é igual ao codomínio): $f(X) = Y$
		\item OBS. toda função função sobrejetora é \textbf{invertível à direita} (e vice-versa). Por quê?
	\end{itemize}

	\begin{adjustbox}{max totalsize={\textwidth}{.65\textheight},center}
	\begin{tikzpicture}
	\centering
	\begin{axis}[
		    axis lines = left,
		    xlabel = $x$,
		    ylabel = {$f(x)$},
		    legend style={at={(1,0.125)},anchor=south east}
		]
		\addplot [
		    domain=-4:2, 
		    samples=100, 
		    color=blue,
		    ]
		    {x+2};
		\addlegendentry{$x + 2$}
		\addplot [
		    domain=-4:2, 
		    samples=100, 
		    color=red,
		]
		{3^x};
		\addlegendentry{$3^x$}
	\end{axis}
	\end{tikzpicture}
	\end{adjustbox}

\end{frame}

\begin{frame}	
	\frametitle{Funções Injetoras, Sobrejetoras e Bijetoras}

	\begin{itemize}
		\item Uma função bijetora (ou bijetiva) é uma onde cada elemento do domínio é mapeado para um elemento do codomínio \textbf{e vice-versa}.
		\item OBS. toda função função bijetora é ao mesmo tempo injetora e sobrejetora. Por quê? Ela é invertível à esquerda ou à direita?
	\end{itemize}

	\begin{adjustbox}{max totalsize={\textwidth}{.65\textheight},center}
	\begin{tikzpicture}
	\centering
	\begin{axis}[
		    axis lines = left,
		    xlabel = $x$,
		    ylabel = {$f(x)$},
		    legend style={at={(0.5,1)},anchor=north}
		]
		\addplot [
		    domain=-4:6, 
		    samples=100, 
		    color=blue,
		    ]
		    {-x};
		\addlegendentry{$-x$}
		\addplot [
		    domain=-4:6, 
		    samples=100, 
		    color=ForestGreen,
		]
		{abs(3*x)};
		\addlegendentry{$|3x|$}
		\addplot [
		    domain=-4:6, 
		    samples=100, 
		    color=red,
		]
		{4*log10(x)};
		\addlegendentry{$4 \log_{10}(x)$}
	\end{axis}
	\end{tikzpicture}
	\end{adjustbox}

\end{frame}

\begin{frame}
	\frametitle{Funções lineares}

	\begin{itemize}
		\item Funções lineares sob várias perspectivas:
		\begin{itemize}
			\item O gráfico de uma função linear no plano cartesiano é uma linha reta
			\item A lei de uma função linear genérica é $f(x) = ax + b$ (com $a,b$ constantes)
			\item Numa função linear, a proporção entre a variação em $y$ e a variação em $x$ é sempre constante $\left(\frac{f(x_1)-f(x_2)}{x_1-x_2} = a\right)$
		\end{itemize}
	\end{itemize}


	\begin{adjustbox}{max totalsize={\textwidth}{.6\textheight},center}
	\begin{tikzpicture}
	\centering
	\begin{axis}[
		    axis lines = left,
		    xlabel = $x$,
		    ylabel = {$f(x)$},
		    legend style={at={(1,1)},anchor=north}
		]
		\addplot [
		    domain=-4:6, 
		    samples=100, 
		    color=blue,
		    ]
		    {-2*x + 3};
		\addlegendentry{$-2x + 3$}
		
		\draw (axis cs:-2,7) -- (axis cs:0,7);
		\draw (axis cs:0,7) -- (axis cs:0,3);

		\draw[fill=white] (axis cs:-2,7) circle (2pt) {(-2,7)};
		\draw[fill=white] (axis cs:0,3) circle (2pt) {(0,3)};

		\node[xshift=-10pt,yshift=-10pt] at (axis cs:-2,7) {(-2,7)};
		\node[xshift=-10pt,yshift=-10pt] at (axis cs:0,3) {(0,3)};

		\draw[decoration={brace,raise=5pt},decorate] (axis cs:-2,7) -- node[yshift=15pt] {$0-(-2)=2$} (axis cs:0,7);
		\draw[decoration={brace,raise=5pt},decorate] (axis cs:0,7) -- node[xshift=35pt] {$7-3=4$} (axis cs:0,3);

	\end{axis}
	\end{tikzpicture}
	\end{adjustbox}

\end{frame}

\end{document}