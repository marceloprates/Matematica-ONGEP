
\documentclass[usenames,dvipsnames,svgnames]{beamer}

\usepackage[utf8]{inputenc}
\usepackage{lmodern}
\usepackage{array} % needed for \arraybackslash
\usepackage{graphicx}
\usepackage{adjustbox} % for \adjincludegraphics
\usepackage{amsmath}
\usepackage{tikz}
\usepackage{tkz-euclide}
\usepackage[portuguese]{babel}
\usepackage{pgfplots}
\usepackage{adjustbox}

\usetikzlibrary{calc,patterns,angles,quotes}

%Information to be included in the title page:
\title{Funções Lineares, Quadráticas}
\author{Matemática}
\institute{ONGEP}
\date{2018}

\begin{document}

\frame{\titlepage}

\begin{frame}	
	\frametitle{Funções Injetoras, Sobrejetoras e Bijetoras}

	\begin{itemize}
		\item Uma função injetora (ou injetiva) é uma onde cada elemento diferente do domínio mapeia para um elemento diferente da imagem: $a \neq b \Rightarrow f(a) \neq f(b)$
		\item OBS. toda função função injetora é \textbf{invertível à esquerda} ($f^{-1}(f(x)) = x$) -- e vice-versa. Por quê?
	\end{itemize}

	\begin{adjustbox}{max totalsize={\textwidth}{.65\textheight},center}
	\begin{tikzpicture}
	\centering
	\begin{axis}[
		    axis lines = left,
		    xlabel = $x$,
		    ylabel = {$f(x)$},
		    legend style={at={(1,0.125)},anchor=south east}
		]
		\addplot [
		    domain=-3:3, 
		    samples=100, 
		    color=blue,
		    ]
		    {x^3};
		\addlegendentry{$x^3$}
		\addplot [
		    domain=-3:3,
		    samples=100, 
		    color=red,
		]
		{4*x^2 - 5};
		\addlegendentry{$4x^2 - 5$}
	\end{axis}
	\end{tikzpicture}
	\end{adjustbox}
\end{frame}

\begin{frame}	
	\frametitle{Funções Injetoras, Sobrejetoras e Bijetoras}

	\begin{itemize}
		\item Uma função sobrejetora (ou sobrejetiva) é uma onde todo elemento no codomínio é ``atingido'' (a imagem da função é igual ao codomínio): $f(X) = Y$
		\item OBS. toda função função sobrejetora é \textbf{invertível à direita} ($f(f^{-1}(x)) = x$) -- e vice-versa. Por quê?
	\end{itemize}

	\begin{adjustbox}{max totalsize={\textwidth}{.65\textheight},center}
	\begin{tikzpicture}
	\centering
	\begin{axis}[
		    axis lines = left,
		    xlabel = $x$,
		    ylabel = {$f(x)$},
		    legend style={at={(1,0.125)},anchor=south east}
		]
		\addplot [
		    domain=-4:2, 
		    samples=100, 
		    color=blue,
		    ]
		    {x+2};
		\addlegendentry{$x + 2$}
		\addplot [
		    domain=-4:2, 
		    samples=100, 
		    color=red,
		]
		{3^x};
		\addlegendentry{$3^x$}
	\end{axis}
	\end{tikzpicture}
	\end{adjustbox}
\end{frame}

\begin{frame}	
	\frametitle{Funções Injetoras, Sobrejetoras e Bijetoras}

	\begin{itemize}
		\item Uma função bijetora (ou bijetiva) é uma onde cada elemento do domínio é mapeado para um elemento do codomínio \textbf{e vice-versa}.
		\item OBS. toda função função bijetora é ao mesmo tempo injetora e sobrejetora. Por quê? Ela é invertível à esquerda ou à direita?
	\end{itemize}

	\begin{adjustbox}{max totalsize={\textwidth}{.65\textheight},center}
	\begin{tikzpicture}
	\centering
	\begin{axis}[
		    axis lines = left,
		    xlabel = $x$,
		    ylabel = {$f(x)$},
		    legend style={at={(0.5,1)},anchor=north}
		]
		\addplot [
		    domain=-4:6, 
		    samples=100, 
		    color=blue,
		    ]
		    {-x};
		\addlegendentry{$-x$}
		\addplot [
		    domain=-4:6, 
		    samples=100, 
		    color=ForestGreen,
		]
		{abs(3*x)};
		\addlegendentry{$|3x|$}
		\addplot [
		    domain=-4:6, 
		    samples=100, 
		    color=red,
		]
		{4*log10(x)};
		\addlegendentry{$4 \log_{10}(x)$}
	\end{axis}
	\end{tikzpicture}
	\end{adjustbox}
\end{frame}

\begin{frame}
	\frametitle{Funções lineares ou do primeiro grau}

	\begin{itemize}
		\item Funções lineares sob várias perspectivas:
		\begin{itemize}
			\item O gráfico de uma função linear no plano cartesiano é uma linha reta
			\item A lei de uma função linear genérica é $f(x) = ax + b$ (com $a,b$ constantes e $a \neq 0$)
			\item Numa função linear, a proporção entre a variação em $y$ e a variação em $x$ é sempre constante $\left(\frac{f(x_1)-f(x_2)}{x_1-x_2} = a\right)$
		\end{itemize}
	\end{itemize}


	\begin{adjustbox}{max totalsize={\textwidth}{.6\textheight},center}
	\begin{tikzpicture}
	\centering
	\begin{axis}[
		    axis lines = left,
		    xlabel = $x$,
		    ylabel = {$f(x)$},
		    legend style={at={(1,1)},anchor=north}
		]
		\addplot [
		    domain=-4:8, 
		    samples=100, 
		    color=blue,
		    ]
		    {-2*x + 3};
		\addlegendentry{$-2x + 3$}
		
		\draw (axis cs:-2,7) -- (axis cs:0,7);
		\draw (axis cs:0,7) -- (axis cs:0,3);
		\draw[fill=white] (axis cs:-2,7) circle (2pt) {(-2,7)};
		\draw[fill=white] (axis cs:0,3) circle (2pt) {(0,3)};
		\node[xshift=-10pt,yshift=-10pt] at (axis cs:-2,7) {(-2,7)};
		\node[xshift=-10pt,yshift=-10pt] at (axis cs:0,3) {(0,3)};
		\draw[decoration={brace,raise=5pt},decorate] (axis cs:-2,7) -- node[yshift=15pt] {$0-(-2)$} (axis cs:0,7);
		\draw[decoration={brace,raise=5pt},decorate] (axis cs:0,7) -- node[xshift=15pt,rotate=-90] {$7-3$} (axis cs:0,3);

		\draw (axis cs:2,-1) -- (axis cs:6,-1);
		\draw (axis cs:6,-1) -- (axis cs:6,-9);
		\draw[fill=white] (axis cs:2,-1) circle (2pt) {(2,-1)};
		\draw[fill=white] (axis cs:6,-9) circle (2pt) {(6,-9)};
		\node[xshift=-10pt,yshift=-10pt] at (axis cs:2,-1) {(2,-1)};
		\node[xshift=-10pt,yshift=-10pt] at (axis cs:6,-9) {(6,-9)};
		\draw[decoration={brace,raise=5pt},decorate] (axis cs:2,-1) -- node[yshift=15pt] {$6-2$} (axis cs:6,-1);
		\draw[decoration={brace,raise=5pt},decorate] (axis cs:6,-1) -- node[xshift=15pt,rotate=-90] {$(-9)-(-1)$} (axis cs:6,-9);

	\end{axis}
	\end{tikzpicture}
	\end{adjustbox}
\end{frame}

\begin{frame}
	\frametitle{Raízes ou soluções de uma função qualquer}

	\begin{itemize}
		\item Frequentemente temos uma função $f(x)$ e desejamos saber para quais valores de $x$ vale $f(x) = 0$
		\item Esses valores de $x$ têm um nome especial: ``raízes'' ou ``soluções'' de $f(x)$
		\item E se desejarmos saber para quais valores de $x$ vale $f(x) = y$ (para algum $y$ arbitrário?)
	\end{itemize}

\end{frame}

\begin{frame}
	\frametitle{Anatomia de uma função linear}

	\begin{itemize}
		\item A constante ``a'' em $f(x) = ax + b$ é chamada ``coeficiente angular'', pois ela determina o ângulo que a reta faz com o eixo $x$.
	\end{itemize}

	\begin{adjustbox}{max totalsize={\textwidth}{.6\textheight},center}
	\begin{tikzpicture}
	\centering
	\begin{axis}[
		    axis lines = left,
		    xlabel = $x$,
		    ylabel = {$f(x)$},
		    legend style={at={(1,1)},anchor=north}
		]
		\addplot [
		    domain=-4:4, 
		    samples=100, 
		    color=BrickRed,
		    ]
		    {-3*x};
		\addlegendentry{$-3x$}
		\addplot [
		    domain=-4:4, 
		    samples=100, 
		    color=BlueViolet,
		    ]
		    {-2*x};
		\addlegendentry{$-2x$}
		\addplot [
		    domain=-4:4, 
		    samples=100, 
		    color=Bittersweet,
		    ]
		    {-x};
		\addlegendentry{$-1x$}
		\addplot [
		    domain=-4:4, 
		    samples=100, 
		    color=blue,
		    ]
		    {0};
		\addlegendentry{$~~0x$}
		\addplot [
		    domain=-4:4, 
		    samples=100, 
		    color=red,
		    ]
		    {x};
		\addlegendentry{$+1x$}
		\addplot [
		    domain=-4:4, 
		    samples=100, 
		    color=ForestGreen,
		    ]
		    {2*x};
		\addlegendentry{$+2x$}
		\addplot [
		    domain=-4:4, 
		    samples=100, 
		    color=WildStrawberry,
		    ]
		    {3*x};
		\addlegendentry{$+3x$}
	\end{axis}
	\end{tikzpicture}
	\end{adjustbox}
\end{frame}

\begin{frame}
	\frametitle{Anatomia de uma função linear}

	\begin{itemize}
		\small{
		\item A constante ``a'' em $f(x) = ax + b$ é chamada ``coeficiente angular'', pois ela determina o ângulo que a reta faz com o eixo $x$.
		\item O coeficiente angular pode ser calculado em qualquer posição da reta se dividirmos a variação em $y$ ($\Delta y$) pela variação em $x$ ($\Delta x$): $a = (\Delta y = y_1 - y_0)/(\Delta x = x_1 - x_0)$
		}
	\end{itemize}

	\begin{adjustbox}{max totalsize={\textwidth}{.6\textheight},center}
	\begin{tikzpicture}
	\centering
	\begin{axis}[
		    axis lines = left,
		    xlabel = $x$,
		    ylabel = {$f(x)$},
		    legend style={at={(1,1)},anchor=north}
		]
		\addplot [
		    domain=-4:4, 
		    samples=100, 
		    color=blue,
		    ]
		    {2*x + 3};
		\addlegendentry{$2x + 3$}
		
		\draw (axis cs:2,3) -- (axis cs:2,7);
		\draw (axis cs:0,3) -- (axis cs:2,3);
		\draw[fill=white] (axis cs:0,3) circle (2pt) {(0,3)};
		\draw[fill=white] (axis cs:2,7) circle (2pt) {(2,7)};
		\node[xshift=-10pt,yshift=10pt] at (axis cs:2,7) {(2,7)};
		\node[xshift=-10pt,yshift=10pt] at (axis cs:0,3) {(0,3)};
		\draw[decoration={brace,raise=5pt,mirror},decorate] (axis cs:2,3) -- node[xshift=15pt,rotate=-90] {$7-3=4$} (axis cs:2,7);
		\draw[decoration={brace,raise=5pt,mirror},decorate] (axis cs:0,3) -- node[yshift=-15pt] {$2-0=2$} (axis cs:2,3);

		\draw (axis cs:0.5,3) arc (0:63:10);
    	\node[xshift=20pt,yshift=7pt] at (axis cs:0,3)  {$\theta$};

    	\draw [->,>=stealth] (axis cs:0.5,4.5) -- (axis cs:-1,8);
    	\node[xshift=20pt,yshift=7pt] at (axis cs:-2,8.5)  {$\tan(\theta \approx 1.1) = 2 = \frac{4}{2}$};

	\end{axis}
	\end{tikzpicture}
	\end{adjustbox}
\end{frame}

\begin{frame}
	\frametitle{Anatomia de uma função linear}

	\begin{itemize}
		\small{
		\item A constante ``b'' em $f(x) = ax + b$ é chamada ``coeficiente linear'' ou ``termo independente'', e determina onde a reta cruza o eixo $y$, porque $f(0) = a \times 0 + b = b$
		\item O coeficiente linear também determina onde a reta cruza o eixo $x$ (ou seja, a raiz ou solução da função linear), pois $f(-b/a) = a \times (-b/a) + b = -b + b = 0$
		}
	\end{itemize}

	\begin{adjustbox}{max totalsize={\textwidth}{.6\textheight},center}
	\begin{tikzpicture}
	\centering
	\begin{axis}[
		    axis lines = left,
		    xlabel = $x$,
		    ylabel = {$f(x)$},
		    legend style={at={(1,1)},anchor=north}
		]
		\addplot [
		    domain=-4:4, 
		    samples=100, 
		    color=blue,
		    ]
		    {2*x + 3};
		\addlegendentry{$2x + 3$}
		
		\draw[dashed] (axis cs:0,-5) -- (axis cs:0,10);
		\draw[dashed] (axis cs:-4,0) -- (axis cs:4,0);

		\draw[fill=white] (axis cs:-3/2,0) circle (2pt) {(-3/2,0)};
		\draw[fill=white] (axis cs:0,3) circle (2pt) {(0,3)};
		\node[xshift=-15pt,yshift=10pt] at (axis cs:-3/2,0) {($-\frac{3}{2}$,0)};
		\node[xshift=-15pt,yshift=10pt] at (axis cs:0,3) {(0,3)};

	\end{axis}
	\end{tikzpicture}
	\end{adjustbox}
\end{frame}

\begin{frame}
	\frametitle{Transformações em funções lineares}

	\begin{itemize}
		\item Observe que podemos mover uma função linear para cima ou para baixo se aumentarmos ou diminuirmos o valor do coeficiente linear (``b''), respectivamente
		\item Demonstração interativa em \url{https://www.openprocessing.org/sketch/568383} \footnote{Arrasta o mouse p/ mover a linha; Arrasta o mouse apertando uma tecla p/ mudar inclinação}
	\end{itemize}

	\begin{adjustbox}{max totalsize={\textwidth}{.55\textheight},center}
	\begin{tikzpicture}
	\centering
	\begin{axis}[
		    axis lines = left,
		    xlabel = $x$,
		    ylabel = {$f(x)$},
		    legend style={at={(1,1)},anchor=north}
		]
		\addplot [
		    domain=-2:2, 
		    samples=100, 
		    color=red,
		    ]
		    {-2*x-1};
		\addlegendentry{$-2x-1$}
		\addplot [
		    domain=-2:2, 
		    samples=100, 
		    color=ForestGreen,
		    ]
		    {-2*x};
		\addlegendentry{$-2x$}
		\addplot [
		    domain=-2:2, 
		    samples=100, 
		    color=blue,
		    ]
		    {-2*x+1};
		\addlegendentry{$-2x + 1$}

		\draw[dashed] (axis cs:0,-5) -- (axis cs:0,10);
		\draw[dashed] (axis cs:-4,0) -- (axis cs:4,0);

	\end{axis}
	\end{tikzpicture}
	\end{adjustbox}
\end{frame}

\begin{frame}
	\frametitle{Transformações em funções lineares}

	\begin{itemize}
		\item Observe também que quando a reta sobe $1$ unidade, é como se ela se deslocasse algumas unidades para a direita. Quantas?
	\end{itemize}

	\begin{adjustbox}{max totalsize={\textwidth}{.6\textheight},center}
	\begin{tikzpicture}
	\centering
	\begin{axis}[
		    axis lines = left,
		    xlabel = $x$,
		    ylabel = {$f(x)$},
		    legend style={at={(1,1)},anchor=north}
		]
		\addplot [
		    domain=-2:2, 
		    samples=100, 
		    color=red,
		    ]
		    {-2*x-1};
		\addlegendentry{$-2x-1$}
		\addplot [
		    domain=-2:2, 
		    samples=100, 
		    color=ForestGreen,
		    ]
		    {-2*x};
		\addlegendentry{$-2x$}
		\addplot [
		    domain=-2:2, 
		    samples=100, 
		    color=blue,
		    ]
		    {-2*x+1};
		\addlegendentry{$-2x + 1$}

		\draw[dashed] (axis cs:0,-5) -- (axis cs:0,10);
		\draw[dashed] (axis cs:-4,0) -- (axis cs:4,0);

	\end{axis}
	\end{tikzpicture}
	\end{adjustbox}
\end{frame}

\begin{frame}
	\frametitle{Raízes de uma função linear}

	\begin{itemize}
		\item Uma função linear tem apenas \textbf{uma} raiz. Por quê?
		\begin{itemize}
			\item Graficamente: porque uma reta só cruza o eixo $x$ uma única vez
			\item Algebricamente: porque $0 = f(x) = ax + b \Leftrightarrow ax + b = 0 \Leftrightarrow x = -b/a$, e $-b/a$ é um único valor (lembrando que no caso das equações do segundo grau temos ${\color{red}\pm} \sqrt{\dots}$)
		\end{itemize}
		\item Podemos escrever qualquer equação linear no seguinte formato: $a(x-r)$, onde $a$ é o coeficiente angular e $r$ é a raiz da equação
		\item Expandindo, temos $a(x-r) = ax - ar$, onde ``$ar$'' equivale ao ``b'' (coeficiente linear) no esquema $f(x) = ax + b$
	\end{itemize}
\end{frame}

\begin{frame}
	\frametitle{Transformações em funções lineares}

	\begin{itemize}
		\item Podemos mover uma reta $1$ unidade para a direita ou para a esquerda subtraindo ou somando $a$ no coeficiente linear, respectivamente
		\item Isso funciona pois $a(x-r) - a = a(x-r-1) = a(x-(r+1))$
		\item Demonstração interativa em \url{https://www.openprocessing.org/sketch/568383} \footnote{Arrasta o mouse p/ mover a linha; Arrasta o mouse apertando uma tecla p/ mudar inclinação}
	\end{itemize}

	\begin{adjustbox}{max totalsize={\textwidth}{.5\textheight},center}
	\begin{tikzpicture}
	\centering
	\begin{axis}[
		    axis lines = left,
		    xlabel = $x$,
		    ylabel = {$f(x)$},
		    legend style={at={(1,1)},anchor=north}
		]
		\addplot [
		    domain=-2:2, 
		    samples=100, 
		    color=red,
		    ]
		    {-2*x-1};
		\addlegendentry{$-2x-1$}
		\addplot [
		    domain=-2:2, 
		    samples=100, 
		    color=ForestGreen,
		    ]
		    {-2*x};
		\addlegendentry{$-2x$}
		\addplot [
		    domain=-2:2, 
		    samples=100, 
		    color=blue,
		    ]
		    {-2*x+1};
		\addlegendentry{$-2x + 1$}

		\draw[dashed] (axis cs:0,-5) -- (axis cs:0,10);
		\draw[dashed] (axis cs:-4,0) -- (axis cs:4,0);

	\end{axis}
	\end{tikzpicture}
	\end{adjustbox}
\end{frame}

\begin{frame}
	\frametitle{Funções quadráticas ou do segundo grau}

	\begin{itemize}
		\small{
		\item Funções quadráticas sob várias perspectivas:
		\begin{itemize}
			\item O gráfico de uma função quadrática no plano cartesiano é uma \emph{parábola} (mas o que é uma parábola? :D)
			\item A lei de uma função quadrática genérica é $f(x) = ax^2 + bx + c$ com $a \neq 0$ (do contrário vira uma equação linear)
			\item A inclinação da curva quadrática (parábola) não é constante como a da reta, mas \textbf{varia linearmente}
		\end{itemize}
		}
	\end{itemize}

	\begin{adjustbox}{max totalsize={\textwidth}{.6\textheight},center}
	\begin{tikzpicture}
	\centering
	\begin{axis}[
		    axis lines = left,
		    xlabel = $x$,
		    ylabel = {$f(x)$},
		    ymin=-2,
		    legend style={at={(0.5,1)},anchor=north}
		]
		\addplot [
		    domain=-5:5, 
		    samples=100, 
		    color=black,
		    ]
		    {x^2};
		\addlegendentry{$x^2$}
		
		\addplot [
		    domain=-5:5,
		    samples=100, 
		    color=red,
		    dashed
		    ]
		    {2*x-1};
		\addplot [
		    domain=-5:5,
		    samples=100, 
		    color=red,
		    dashed
		    ]
		    {4*x-4};
		\addplot [
		    domain=-5:5,
		    samples=100, 
		    color=red,
		    dashed
		    ]
		    {6*x-9};
		\addplot [
		    domain=-5:5,
		    samples=100, 
		    color=red,
		    dashed
		    ]
		    {0};
		\addplot [
		    domain=-5:5,
		    samples=100, 
		    color=red,
		    dashed
		    ]
		    {-2*x-1};
		\addplot [
		    domain=-5:5,
		    samples=100, 
		    color=red,
		    dashed
		    ]
		    {-4*x-4};
		\addplot [
		    domain=-5:5,
		    samples=100, 
		    color=red,
		    dashed
		    ]
		    {-6*x-9};

	\end{axis}
	\end{tikzpicture}
	\end{adjustbox}
\end{frame}

\begin{frame}
	\frametitle{O que é uma parábola?}

	\begin{itemize}
		\small{
		\item Uma parábola é um conjunto de pontos que estão à mesma distância de um {\color{blue}ponto focal} e de uma {\color{red}reta} (chamada {\color{red}diretriz})
		\item Demonstração interativa: \url{www.openprocessing.org/sketch/567057} \footnote{arrasta o mouse p/ mudar o ângulo da diretriz; arrasta o mouse apertando uma tecla do teclado p/ movimentar o ponto focal}
		}
	\end{itemize}

	\begin{adjustbox}{max totalsize={\textwidth}{.6\textheight},center}
	\begin{tikzpicture}
	\centering
	\begin{axis}[
		    axis lines = left,
		    xlabel = $x$,
		    ylabel = {$f(x)$},
		    ymin=-5,
		    legend style={at={(0.5,1)},anchor=north}
		]
		\addplot [
		    domain=-5:5, 
		    samples=100, 
		    color=black,
		    ]
		    {x^2/4};
		\addlegendentry{$y = \frac{1}{4}x^2$}

		\draw[color=blue,fill=white] (axis cs:0,1) circle (2pt) {(0,1)};
		\node[xshift=0pt,yshift=10pt] at (axis cs:0,1) {(0,1)};

		\addplot [
		    domain=-5:5,
		    samples=100, 
		    color=red
		    ]
		    {-1};
		\addlegendentry{$y = -1$}

	\end{axis}
	\end{tikzpicture}
	\end{adjustbox}
\end{frame}

\begin{frame}
	\frametitle{Por que o gráfico de uma função quadrática é parabólico?}

	\begin{itemize}
		\item Considere a parábola do exemplo anterior, com foco $F = (0,1)$ e diretriz $y = -1$
		\item A distância de um ponto $(x,y)$ até o foco é $\sqrt{(x-0)^2 + (y-1)^2} = \sqrt{x^2 + (y-1)^2}$
		\item A distância de um ponto $(x,y)$ até a diretriz é a ``altura'' do ponto relativo a ela: $y-(-1) = y+1$
		\item A parábola é o conjunto dos pontos para os quais as distâncias são iguais: $\sqrt{x^2 + (y-1)^2} = y+1$
	\end{itemize}

	\begin{equation}
	\begin{aligned}
		\sqrt{x^2 + (y-1)^2} = y+1 \\
		x^2 + y^2 -2y + 1 = y^2 + 2y + 1 \\
		x^2 -2y = 2y \\
		x^2 = 4y \\
		\frac{1}{4}x^2 = y
	\end{aligned}
	\end{equation}
\end{frame}

\begin{frame}
	\frametitle{Raízes ou soluções de uma função quadrática}

	\begin{itemize}
		\item Toda função quadrática pode ser escrita no formato $f(x) = a(x-r_1)(x-r_2)$, onde $r_1$ e $r_2$ são as raízes (ou soluções) de $f(x)$. Por que isso funciona?
	\end{itemize}

\end{frame}

\begin{frame}
	\frametitle{Anatomia da fórmula quadrática (ou de Bháskara)}

	\begin{itemize}
		\item Como encontrar essas raízes?
	\end{itemize}

	\begin{equation}
		r_1, r_2 = \frac{-b \pm \sqrt{b^2-4ac}}{2a}
	\end{equation}

	\begin{adjustbox}{max totalsize={\textwidth}{.6\textheight},center}
	\begin{tikzpicture}
	\centering
	\begin{axis}[
		    axis lines = left,
		    xlabel = $x$,
		    ylabel = {$f(x)$},
		    legend style={at={(0.5,1)},anchor=north}
		]
		\addplot [
		    domain=1:4, 
		    samples=100, 
		    color=black,
		    ]
		    {x^2 -5*x + 6};
		\addlegendentry{$f(x) = x^2 -5x + 6$}

		\draw[color=blue,fill=white] (axis cs:2,0) circle (2pt) {(2,0)};
		\node[xshift=5pt,yshift=12pt] at (axis cs:2,0) {(2,0)};

		\draw[color=blue,fill=white] (axis cs:3,0) circle (2pt) {(3,0)};
		\node[xshift=-5pt,yshift=12pt] at (axis cs:3,0) {(3,0)};

		\draw[dashed] (axis cs:1,0) -- (axis cs:4,0);

	\end{axis}
	\end{tikzpicture}
	\end{adjustbox}

\end{frame}

\begin{frame}
	\frametitle{Anatomia da fórmula quadrática (ou de Bháskara)}

	\begin{itemize}
		\item Pergunta: se multiplicarmos uma função $f(x) = ax^2 + bx +c$ por alguma constante, as raízes mudam?
	\end{itemize}

	\begin{equation}
		r_1, r_2 = \frac{-b \pm \sqrt{b^2-4ac}}{2a}
	\end{equation}

	\begin{adjustbox}{max totalsize={\textwidth}{.6\textheight},center}
	\begin{tikzpicture}
	\centering
	\begin{axis}[
		    axis lines = left,
		    xlabel = $x$,
		    ylabel = {$f(x)$},
		    legend style={at={(0.5,1)},anchor=north}
		]
		\addplot [
		    domain=1:4, 
		    samples=100, 
		    color=black,
		    ]
		    {x^2 -5*x + 6};
		\addlegendentry{$f(x) = x^2 -5x + 6$}

		\draw[color=blue,fill=white] (axis cs:2,0) circle (2pt) {(2,0)};
		\node[xshift=5pt,yshift=12pt] at (axis cs:2,0) {(2,0)};

		\draw[color=blue,fill=white] (axis cs:3,0) circle (2pt) {(3,0)};
		\node[xshift=-5pt,yshift=12pt] at (axis cs:3,0) {(3,0)};

		\draw[dashed] (axis cs:1,0) -- (axis cs:4,0);

	\end{axis}
	\end{tikzpicture}
	\end{adjustbox}

\end{frame}

\begin{frame}
	\frametitle{Anatomia da fórmula quadrática (ou de Bháskara)}

	\begin{itemize}
		\item E se multiplicarmos por um número negativo?
	\end{itemize}

	\begin{equation}
		r_1, r_2 = \frac{-b \pm \sqrt{b^2-4ac}}{2a}
	\end{equation}

\end{frame}

\begin{frame}
	\frametitle{Anatomia da fórmula quadrática (ou de Bháskara)}

	\begin{itemize}
		\item E se multiplicarmos por um número negativo?
	\end{itemize}

	\begin{equation}
		r_1, r_2 = \frac{-b \pm \sqrt{b^2-4ac}}{2a}
	\end{equation}

	\begin{adjustbox}{max totalsize={\textwidth}{.6\textheight},center}
	\begin{tikzpicture}
	\centering
	\begin{axis}[
		    axis lines = left,
		    xlabel = $x$,
		    ylabel = {$f(x)$},
		    legend style={at={(0.5,0.25)},anchor=north}
		]
		\addplot [
		    domain=1:4, 
		    samples=100, 
		    color=black,
		    ]
		    {-x^2 +5*x - 6};
		\addlegendentry{$f(x) = -x^2 +5x - 6$}

		\draw[color=blue,fill=white] (axis cs:2,0) circle (2pt) {(2,0)};
		\node[xshift=5pt,yshift=12pt] at (axis cs:2,0) {(2,0)};

		\draw[color=blue,fill=white] (axis cs:3,0) circle (2pt) {(3,0)};
		\node[xshift=-5pt,yshift=12pt] at (axis cs:3,0) {(3,0)};

		\draw[dashed] (axis cs:1,0) -- (axis cs:4,0);

	\end{axis}
	\end{tikzpicture}
	\end{adjustbox}

\end{frame}

\begin{frame}
	\frametitle{Anatomia da fórmula quadrática (ou de Bháskara)}

	\begin{itemize}
		\item Pergunta: como descobrimos o mínimo (o ponto mais baixo) ou o máximo (o ponto mais alto) da função?
	\end{itemize}

	\begin{equation}
		r_1, r_2 = \frac{-b \pm \sqrt{b^2-4ac}}{2a}
	\end{equation}

	\begin{adjustbox}{max totalsize={\textwidth}{.6\textheight},center}
	\begin{tikzpicture}
	\centering
	\begin{axis}[
		    axis lines = left,
		    xlabel = $x$,
		    ylabel = {$f(x)$},
		    ymin = -0.5,
		    legend style={at={(0.5,1)},anchor=north}
		]
		\addplot [
		    domain=1:4, 
		    samples=100, 
		    color=black,
		    ]
		    {x^2 -5*x + 6};
		\addlegendentry{$f(x) = x^2 -5x + 6$}

		\draw[dashed] (axis cs:1,0) -- (axis cs:4,0);
		\draw[dashed] (axis cs:5/2,-1) -- (axis cs:5/2,2);

		\draw[color=blue,fill=white] (axis cs:2,0) circle (2pt) {(2,0)};
		\node[xshift=5pt,yshift=12pt] at (axis cs:2,0) {(2,0)};

		\draw[color=blue,fill=white] (axis cs:3,0) circle (2pt) {(3,0)};
		\node[xshift=-5pt,yshift=12pt] at (axis cs:3,0) {(3,0)};

		\draw[color=blue,fill=white] (axis cs:5/2,-1/4) circle (2pt) {(1/2,-1/4)};
		\node[xshift=30pt,yshift=-10pt] at (axis cs:5/2,-1/4) {(2.5,-0.25)};

	\end{axis}
	\end{tikzpicture}
	\end{adjustbox}

\end{frame}

\begin{frame}
	\frametitle{Anatomia da fórmula quadrática (ou de Bháskara)}

	\begin{itemize}
		\item Pergunta: e quando o gráfico não encosta no eixo x? Quais são as raízes?
	\end{itemize}

	\begin{equation}
		r_1, r_2 = \frac{-b \pm \sqrt{b^2-4ac}}{2a}
	\end{equation}

	\begin{adjustbox}{max totalsize={\textwidth}{.6\textheight},center}
	\begin{tikzpicture}
	\centering
	\begin{axis}[
		    axis lines = left,
		    xlabel = $x$,
		    ylabel = {$f(x)$},
		    ymin = -0.5,
		    legend style={at={(0.5,1)},anchor=north}
		]
		\addplot [
		    domain=-2:2, 
		    samples=100, 
		    color=black,
		    ]
		    {x^2 + 1};
		\addlegendentry{$f(x) = x^2 + 1$}

		\draw[dashed] (axis cs:-2,0) -- (axis cs:2,0);
		\draw[dashed] (axis cs:0,-0.5) -- (axis cs:0,5);

	\end{axis}
	\end{tikzpicture}
	\end{adjustbox}

\end{frame}

\begin{frame}
	\frametitle{Anatomia da fórmula quadrática (ou de Bháskara)}

	\begin{itemize}
		\item Pergunta: e quando o gráfico não encosta no eixo x? Quais são as raízes? $i$ e $-i$, pois $(x-i)(x+i) = x^2 + ix -ix -i^2 = x^2 + 1$
	\end{itemize}

	\begin{equation}
		r_1, r_2 = \frac{-b \pm \sqrt{b^2-4ac}}{2a}
	\end{equation}

	\pgfplotsset{plot coordinates/math parser=false}
	\begin{adjustbox}{max totalsize={\textwidth}{.6\textheight},center}
	\begin{tikzpicture}
	\centering
	\begin{axis}[
		    axis lines = left,
		    ylabel = {$\mathcal{R}(x)$},
		    xlabel = {$\mathcal{I}(x)$},
		    zlabel = {$\mathcal{R}(f(x))$},
		    legend style={at={(0,1)},anchor=north}
		]

		\addplot3[
			domain=-1:1,
			surf,
			scale only axis
		]
		{ -x^2 + y^2 + 1 }; % (a+bi)^2 + 1 = a² + 2abi -b² + 1
		\addlegendentry{$f(x) = x^2 + 1$}

		\draw[dashed] (axis cs:-1,0,0) -- (axis cs:1,0,0);
		\draw[dashed] (axis cs:0,-1,0) -- (axis cs:0,1,0);

		\draw[color=blue,fill=white] (axis cs:1,0,0) circle (4pt);
		\node[xshift=-10pt,yshift=30pt] at (axis cs:1,0,0) {(i,0)};

		\draw[color=blue,fill=white] (axis cs:-1,0,0) circle (4pt);
		\node[xshift=-5pt,yshift=20pt] at (axis cs:-1,0,0) {(-i,0)};

	\end{axis}
	\end{tikzpicture}
	\end{adjustbox}

\end{frame}


\end{document}